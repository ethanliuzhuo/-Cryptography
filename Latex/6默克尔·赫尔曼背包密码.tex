\documentclass{article}
\usepackage[UTF8]{ctex}
\usepackage[utf8]{inputenc} % allow utf-8 input
\usepackage[T1]{fontenc}    % use 8-bit T1 fonts
\usepackage{hyperref}       % hyperlinks
\usepackage{url}            % simple URL typesetting
\usepackage{booktabs}       % professional-quality tables
\usepackage{amsfonts}       % blackboard math symbols
\usepackage{nicefrac}       % compact symbols for 1/2, etc.
\usepackage{microtype}      % microtypography
\usepackage{lipsum}
\usepackage{geometry}
\usepackage{graphicx} %插入图片的宏包
\usepackage{float} %设置图片浮动位置的宏包
\usepackage{subfigure} %插入多图时用子图显示的宏包
\usepackage{listings}
\usepackage{soul}


\geometry{a4paper,scale=0.8}
\date{}

\usepackage{listings}
\usepackage{color}

\definecolor{dkgreen}{rgb}{0,0.6,0}
\definecolor{gray}{rgb}{0.5,0.5,0.5}
\definecolor{mauve}{rgb}{0.58,0,0.82}

\lstset{frame=tb,
  language=Python,
  aboveskip=3mm,
  belowskip=3mm,
  showstringspaces=false,
  columns=flexible,
  basicstyle={\small\ttfamily},
  numbers=none,
  numberstyle=\tiny\color{gray},
  keywordstyle=\color{blue},
  commentstyle=\color{dkgreen},
  stringstyle=\color{mauve},
  breaklines=true,
  breakatwhitespace=true,
  tabsize=3
}

\title{默克尔-赫尔曼背包密码}


\author{
Merkle-Hellman Knapsack Cipher\\
 刘卓\\
 \texttt{ } \\
}

\begin{document}
\maketitle

\section{欧几里得算法-伯利坎普}
欧几里得算法(Euclidean Algorithm),也称为辗转相除法。是计算两个给定整数A和B的最大公因数d的过程。其中$A > B$。

伯利坎普对欧几里得算法略微进行修改,可以计算得到B的逆 mod A( inverse of B mod A)。也称伯利坎普算法(Berlekamp Algorithm).

\textbf{算法步骤:}
\begin{enumerate}
\item 设$r_{-2}=A, r_{-1}=B, p_{-2}=0, p_{-1}=1, q_{-2}=1, q_{-1}=0$;
\item 对于 $k = 0,1,2,....$, 逐个计算;
$$
\begin{array}{l}
a_{k} = r_{k-2} \div r_{k-1}\mbox{(取整)}\\
r_{k-2}=a_{k} r_{k-1}+r_{k} \\
p_{k}=a_{k} p_{k-1}+p_{k-2} \\
q_{k}=a_{k} q_{k-1}+q_{k-2}
\end{array}
$$
\item 当$r_{n}=0$, 停止计算;

然后$r_{n-1}=\operatorname{gcd}(A, B)$。此外,在这种情况下 $\operatorname{gcd}(A, B)=1$,并且
$$
B \cdot(-1)^{n} p_{n-1}=1+A \cdot(-1)^{n} q_{n-1}
$$
\end{enumerate}

如果$\operatorname{gcd}(A, B) \ne 1$,则$B^{-1}$不存在。

\textbf{例1}

计算115的逆并 mod 12659.

解:

$$
\begin{array}{|c|c|c|c|c|}
\hline \mathrm{k} & \mathrm{r} & \mathrm{a} & \mathrm{p} & \mathrm{q} \\
\hline-2\mbox{(初始)} & {\color{red}12659(A)} & & 0\mbox{(初始)} & 1\mbox{(初始)} \\
\hline-1\mbox{(初始)} & {\color{red}115(B)} & & 1\mbox{(初始)} & 0\mbox{(初始)} \\
\hline 0 & 9 & 110 & 110 & 1 \\
\hline 1 & 7 & 12 & 1321 & 12 \\
\hline 2 & 2 & 1 & 1431 & 13 \\
\hline 3 & 1 & 3 & {\color{green}5614} & 51 \\
\hline 4(\mbox{停止计算}) & 0 & 2 & {\color{red}12659(A)} & {\color{red}115(B)} \\
\hline
\end{array}
$$

$B^{-1} = (-1)^{n}P_{n-1}=(-1)^{4}5614 = 5614$ \quad\quad\quad\quad\quad\quad\quad\quad\quad\quad\quad\quad\quad\quad\quad\quad\quad\quad\quad\quad\quad   \Box

\section{子集和问题}
子集和问题(Subset-Sum Problem)。给定一串递增数列$a_{1}<a_{2}< \cdots < a_{n}$和一个目标数字M。问递增数列是否存在某个非空子集,使得子集内中的数字和为M。即
$$x_1a_1+x_2a_2+\cdots+x_na_n = M$$
其中$x_i$ 为0或1,$x_i$ 不全为0



定义:\textbf{超级递增序列(a super-increasing sequence)}是序列中每一个数都大于它之前所有数的总和。即:
$$
a_{k}>\sum_{i=0}^{k-1} a_{i}
$$

注意,有时候存在多组解。

~\\

\textbf{例2}

数列$[3,5,11,23,51],M=67$,尝试解决SSP问题。

解:

$$67=51+11+5=a_2+a_3+a_5$$
$$x=[0,1,1,0,1]$$

即存在一组序列x使该问题可以被解决。\quad\quad\quad\quad\quad\quad\quad\quad\quad\quad\quad\quad\quad\quad\quad\quad\quad\quad\quad\quad\quad   \Box

~\\

\textbf{例3}

数列$[13,18,35,72,155,301,595],M=1003$,尝试解决SSP问题。

解:

\begin{eqnarray}   
\label{eq}
1003&=&595 + 408 \nonumber \\ 
&=&  595+301+107\nonumber \\ 
&=& 595+301+72+35\nonumber \\ 
&=& a_3+a_4+a_6+a_7\nonumber \\ 
x&=&[0,0,1,1,0,1,1]\nonumber \\ 
\nonumber 
\end{eqnarray}

即存在一组序列x使该问题可以被解决。\quad\quad\quad\quad\quad\quad\quad\quad\quad\quad\quad\quad\quad\quad\quad\quad\quad\quad\quad\quad\quad   \Box

\clearpage

\section{默克尔-赫尔曼背包密码}

\textbf{加密过程}:

\begin{enumerate}
\item {\color{red}Bob}作为收信人(receiver)
\begin{itemize}
\item 选择一个超级递增序列(a super-increasing sequence) a = $(a_1,a_2,...,a_n)$
\item 选择一个质数$p > \sum_{i=1}^{n} a_i$ 
\item 选择一个加密因子A,A必须 满足$2 \leq A \leq p-1$
\item 保留为秘密,不公开
\end{itemize}
\item {\color{red}Bob}计算出$b_i$序列,$b_i = A \cdot a_i \bmod p$,然后把序列发送给{\color{blue}Alice}作为发信人(sender)
\item {\color{blue}Alice}收到{\color{red}Bob}发送的序列后,需要利用这串序列加密明文,假设明文为二进制 信息$x_1x_2x_3...x_n$,由n个0或1组成。$x_i$为ASCII二进制的所在位置,计算
$$C = x_1b_1+x_2b_2+...+x_nb_n$$
然后将C发送给{\color{red}Bob}。注意:如果字符串长度少于n,则补齐;长度大于n,则截取分段发送。
\item {\color{red}Bob}收到{\color{blue}Alice}发送的信息后,开始解密。先计算$M = A^{-1}C \bmod p$,然后解决SSP问题
$$x_1a_1+x_2a_2+\cdots+x_na_n = M$$

计算得出序列X,即可得知二进制信息$x_1x_2...x_n$,再通过ASCII表格翻译成字符。
\end{enumerate}

~\\

\textbf{例4}

{\color{red}Bob}选择了一个序列
$$
\left(a_{1}, a_{2}, \ldots, a_{8}\right)=(2,5,9,22,47,99,203,409)
$$
和质数$p=997$和加密因子 $A=60$.

\begin{enumerate}
\item 则序列b是
$$b = 60 \times (2,5,9,22,47,99,203,409) \bmod 997
= (120, 300, 540, 323, 826, 955, 216, 612)
$$

{\color{blue}Alice}收到序列b后,加密信息"$b$",$b$的ASCII二进制代码为:01100010.

\item 则C为:
$$C = 0 \cdot b_1 + 1\cdot b_2 + \cdots +0 \cdot b_8 =1056$$

\item {\color{red}Bob}收到C后,尝试解密。
使用Python代码计算$A^{-1}$,方法为:
$$\hl{\textit{ pow(A, -1, p)}}$$
$$A^{-1} = \textit{ pow(60, -1, 997)} = 781$$

\item 计算$M =A^{-1} \cdot C \bmod p $
$$M  = 781 \cdot 1056 \bmod 997 = 217$$

\item 解决SSP问题
$$217 = 5+9+203 = a_2 + a_3 + a_7$$
$$X = (0,1,1,0,0,0,1,0) $$

\item 根据二进制代码转化为字符\textbf{'b'}
\end{enumerate}

\clearpage

\textbf{例5}

\begin{lstlisting}
Plaintext = 'AGREE' #明文

a = [2,5,9,22,47,99,203,409] #Bob设置超级递增数列
p = 997  #Bob设置质数
A = 60 #Bob设置加密因子

b  = [i*A%p for i in a] #求b数列,发送给Alice

A_inv = pow(A,-1,p) #Bob求A的逆

def subset_sum(t,s,x,n,M,a,X):
    """
    子集和问题求解,返回空集合或者序列X

    Parameters
    ----------
    t : int
        递归深度.
    s : int
        子集和.
    x : dict
        判断列表,状态为1或0.
    n : int
        序列长度.
    M : int
        序列长度.
    a : list
        集合b,使得x_1b_1+x_2b_2+...+x_nb_n=M.
    X : list:
        空list,储存结果

    Returns
    -------
    X: list
    0 或 1.
    """
    if t == n:
        if s == M: 
            for i in range(0,n):
                if x[i] != 0: 
                    X += [1] #取b_i
                else:
                    X += [0] #不取b_i
            
    else:
        s = s+a[t]
        x[t] = a[t]
        subset_sum(t+1,s,x,n,M,a,X)
        s = s-a[t]         #回溯之前还原
        x[t] = 0
        subset_sum(t+1,s,x,n,M,a,X)
        
    return X

        
for i in Plaintext:
    """Alice 加密过程"""
    c_int = ord(i)
    c_bin = bin(c_int) #将字符转化为二进制
    c = 0
    
    if len(c_bin[2:]) != 8:
        c_bin = '0'*(8-len(c_bin[2:])) + c_bin[2:] #二进制转为8位二进制,与a长度相同
    else:
        c_bin = c_bin[2:]
        
    for j in range(8):
        try:
            # print(c_bin[j],b[j],j)
            c += int(c_bin[j])*b[j] #计算c,然后发送给Bob
        except:
            pass
    # print(c)
     """Bob 解密过程"""
    M = A_inv * c%p
    # print(M)
    
    t = 0            #递归深度
    s = 0            #子集和
    x = {}           #判断列表,状态为1或0
    n = len(a)
    X = []
    
    X = subset_sum(t,s,x,n,M,a,X) #解决子集和问题
    # print(X)
    
    bin_X = ''
    for k in X:
        bin_X += str(k)
    print(chr(int(bin_X,2))) #二进制转化ASCII字符
\end{lstlisting}

输出:AGREE

\end{document}

