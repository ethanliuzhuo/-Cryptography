\documentclass{article}
\usepackage[UTF8]{ctex}
\usepackage[utf8]{inputenc} % allow utf-8 input
\usepackage[T1]{fontenc}    % use 8-bit T1 fonts
\usepackage{hyperref}       % hyperlinks
\usepackage{url}            % simple URL typesetting
\usepackage{booktabs}       % professional-quality tables
\usepackage{amsfonts}       % blackboard math symbols
\usepackage{nicefrac}       % compact symbols for 1/2, etc.
\usepackage{microtype}      % microtypography
\usepackage{lipsum}
\usepackage{geometry}
\usepackage{graphicx} %插入图片的宏包
\usepackage{float} %设置图片浮动位置的宏包
\usepackage{subfigure} %插入多图时用子图显示的宏包
\usepackage{listings}

\geometry{a4paper,scale=0.8}
\date{}

\usepackage{listings}
\usepackage{color}

\definecolor{dkgreen}{rgb}{0,0.6,0}
\definecolor{gray}{rgb}{0.5,0.5,0.5}
\definecolor{mauve}{rgb}{0.58,0,0.82}

\lstset{frame=tb,
  language=Python,
  aboveskip=3mm,
  belowskip=3mm,
  showstringspaces=false,
  columns=flexible,
  basicstyle={\small\ttfamily},
  numbers=none,
  numberstyle=\tiny\color{gray},
  keywordstyle=\color{blue},
  commentstyle=\color{dkgreen},
  stringstyle=\color{mauve},
  breaklines=true,
  breakatwhitespace=true,
  tabsize=3
}

\title{矩阵换位}


\author{
Rectangular Transposition\\
 刘卓\\
 \texttt{ } \\
}

\begin{document}
\maketitle

\section{普费尔密码}

普费尔密码(Playfair Cipher)由查尔斯·惠斯通(Charles Wheatstone)发明,里昂·普费尔(Lyon Playfair)推广, 因此得名普莱费尔密码。普费尔密码在第二次布尔战争和第一次世界大战中被英军广泛使用,之后的第二次世界大战中澳大利亚人也使用它。

~\\

\textbf{加密步骤}

\subsection{确定密钥}

~\\

\textbf{例1}

假设普费尔密码矩阵使用密钥(Keyword): $DIVERGENT$

将密钥放置在一个$5 \times 5$ 矩阵中,如果遇到相同字母,则跳过。比如$DIVERGENT$,填充时第二个字母{\color{red}E}跳过忽略。通常字母{\color{red}I}和字母{\color{red}J}放在一个框框内。剩下的空则按拉丁字母表挨个填充,遇到已有字母则跳过。

$$
\begin{array}{|c|c|c|c|c|}
\hline {\color{red}D} &  {\color{red}I / J} &  {\color{red}V} &  {\color{red}E} &  {\color{red}R} \\
\hline  {\color{red}G} &  {\color{red}N} &  {\color{red}T} & A & B \\
\hline C & F & H & K & L \\
\hline M & O & P & Q & S \\
\hline U & W & X & Y & Z \\
\hline
\end{array}
$$ 

\subsection{将要加密的讯息分成两个一组}

\begin{itemize}
\item 如果字符串中有重复的字母,则将字母$\ X $插入进重复的字母
\item 如果字符串长度是奇数,则在字符串末尾加一个字母$\ Q $
\end{itemize}

~\\

\textbf{例2}

明文$PLAN$, 则划分为$PL \| AN$ 

明文$CHEEG$, 则划分为$CH \| EX \| EG$ 

明文$ACT$, 则划分为$AC \| TQ$  \qquad\qquad\qquad\qquad\qquad\qquad\qquad\qquad\qquad\qquad\qquad\qquad \Box

\subsection{加密}
\begin{itemize}
\item 若两个字母在同一列,取这两个字母\textbf{右}方的字母(若字母在最右方则取最左方的字母)。
\item 若两个字母在同一行,取这两个字母\textbf{下}方的字母(若字母在最下方则取最上方的字母)。
\item 若两个字母不在同一横列或同一直行,则在矩阵中找出另外两个字母,使这四个字母成为一个长方形的四个角,取对应行的字母。
\end{itemize}

~\\

\textbf{例3}

根据例1表格:

%$$
%\begin{array}{|c|c|c|c|c|}
%\hline D &  I / J & V&  E &  R\\
%\hline  G&  N &  T & A & B \\
%\hline C & F & H & K & L \\
%\hline M & O & P & Q & S \\
%\hline U & W & X & Y & Z \\
%\hline
%\end{array}
%$$

规则1中,如字符${\color{red}GT}$ 则加密为${\color{blue}NA}$;如字符${\color{red}TG}$ 则加密为${\color{blue}AN}$;如字符${\color{red}NB}$ 则加密为${\color{blue}TG}$(因B在最右,故取最左边的字母G);

$$
\begin{array}{|c|c|c|c|c|}
\hline &  & &   &   \\
\hline {\color{red}G} &  {\color{blue}N} &  {\color{red}T} & {\color{blue}A} &  \\
\hline &  & &   &   \\
\hline &  & &   &   \\
\hline &  & &   &   \\
\hline
\end{array}
$$

 
规则2中,如字符${\color{red}MG}$ 则加密为${\color{blue}UC}$;如字符${\color{red}DC}$ 则加密为${\color{blue}GM}$;如字符${\color{red}GU}$ 则加密为${\color{blue}CD}$(因U在最下,故取最上边的字母D);

$$
\begin{array}{|c|c|c|c|c|}
\hline &  & &   &   \\
\hline {\color{red}G}&  & &   &   \\
\hline {\color{blue}C}&  & &   &   \\
\hline {\color{red}M}&  & &   &   \\
\hline {\color{blue}U}&  & &   &   \\
\hline
\end{array}
$$

规则3中,如字符${\color{red}AO}$ 则加密为${\color{blue}NQ}$;如字符${\color{red}OA}$ 则加密为${\color{blue}QN }$;如字符${\color{red}NZ}$ 则加密为${\color{blue}BW}$;
 
$$
\begin{array}{|c|c|c|c|c|}
\hline  &  & &   &   \\
\hline  & {\color{blue}N} & & {\color{red}A}  & \\
\hline  &  & &   &  \\
\hline  & {\color{red}O} & &  {\color{blue}Q} &  \\
\hline  &  & &   &  \\
\hline
\end{array}
$$


一共拥有$25!$的密钥可能性。


\section{ADFGVX密码}
ADFGVX密码由德国上校弗里茨·内贝尔(Fritz Nebel)发明。ADFGVX密码和Playfair密码相似,也是矩阵换位密码的一种。

\subsection{加密步骤}
\begin{enumerate}
\item 确定一个密钥,构建一个$6 \times 6$ ADFGVX矩阵,输入26个拉丁字母和10个数字;
\item 将明文中的每个字母转换为其在ADFGVX中的坐标表。坐标的顺序为(行索引、列索引);
\item 将转换后的文本(从左到右逐行)重新排列为一种含有n列的表,并使用长度为n的选定排列对这些列进行排列;
\item 从上到下逐列读取已排列的表以获得密文;
\end{enumerate}

密钥空间 $=36!$

~\\

\textbf{例4}

如密钥是SUMMER,则ADFGVX矩阵是:
$$
\begin{array}{|c|c|c|c|c|c|c|}
\hline & \mathbf{A} & \mathbf{D} & \mathbf{F} & \mathbf{G} & \mathbf{V} & \mathbf{X} \\
\hline \mathbf{A} &{\color{blue}\mathbf{S}} & {\color{blue}\mathrm{U}} & {\color{blue}\mathrm{M}} & {\color{blue}\mathrm{E}} & {\color{blue}\mathrm{R}} & \mathrm{A} \\
\hline \mathbf{D} &\mathbf{B} & \mathrm{C} & \mathrm{D} & \mathrm{F} & \mathrm{G} & \mathrm{H} \\
\hline \mathbf{F}&\mathbf{I} &  \mathrm{J} & \mathrm{K} & \mathrm{L} & \mathrm{N} & \mathrm{O} \\
\hline \mathbf{G}&\mathrm{P} & \mathrm{Q} & \mathrm{T} & \mathrm{V} & \mathrm{W}& \mathrm{X}  \\
\hline \mathbf{V}&\mathbf{Y} & \mathrm{Z} & 0& \mathrm{1} & \mathrm{2} & \mathrm{3} \\
\hline \mathbf{X}&\mathrm{4} & \mathrm{5} & \mathrm{6} & 7 & \mathrm{8} & 9 \\
\hline
\end{array}\\
$$


~\\

\textbf{例5}

使用列表\textbf{ $8\ 4\ 3\ 2\ 7\ 6\ 1\ 5$ }和如下表格加密\textit{$“from\ one\ day\ to\ another\ in\ battle"$}
$$
\begin{array}{|c|c|c|c|c|c|c|}
\hline & \mathbf{A} & \mathbf{D} & \mathbf{F} & \mathbf{G} & \mathbf{V} & \mathbf{X} \\
\hline \mathbf{A} & \mathrm{i} & \mathrm{w} & \mathrm{o} & \mathrm{u} & \mathrm{l} & \mathrm{d} \\
\hline \mathbf{D} & \mathrm{e} & \mathrm{f} & \mathrm{r} & \mathrm{y} & 0 & \mathrm{a} \\
\hline \mathbf{F} & 9 & \mathrm{~b} & \mathrm{c} & 1 & \mathrm{~g} & \mathrm{~h} \\
\hline \mathrm{G} & \mathrm{j} & \mathrm{k} & 2 & \mathrm{~m} & \mathrm{n} & 7 \\
\hline \mathbf{V} & \mathrm{p} & 3 & \mathrm{q} & \mathrm{s} & 6 & \mathrm{t} \\
\hline \mathrm{X} & 4 & \mathrm{v} & \mathrm{x} & 5 & \mathrm{z} & 8 \\
\hline
\end{array}\\
$$

解:

将明文中的每个字母转换为其在ADFGVX中的坐标表

$$
\begin{array}{|c|c|c|c|c|c|c|c|c|c|}
\hline \mathrm{f} & \mathrm{r} & \mathrm{o} & \mathrm{m} & \mathrm{o} & \mathrm{n} & \mathrm{e} & \mathrm{d} & \mathrm{a} & \mathrm{y} \\
\hline DD& DF& AF& GG& AF& GV& DA& AX& DX&DG \\
\hline \mathrm{t} & \mathrm{o} & \mathrm{a} & \mathrm{n} & \mathrm{o} & \mathrm{t} & \mathrm{h} & \mathrm{e} & \mathrm{r} & \\
\hline VX& AF& DX& GV& AF& VX& FX& DA& DF& \\
\hline \mathrm{i} & \mathrm{n} & \mathrm{b} & \mathrm{a} & \mathrm{t} & \mathrm{t} & \mathrm{l} & \mathrm{e} & & \\
\hline AA& GV& FD& DX& VX& VX& AV& DA& & \\
\hline
\end{array}\\
$$

然后逐行填入列表\textbf{ $8\ 4\ 3\ 2\ 7\ 6\ 1\ 5$ }所形成的矩阵中。

$$
\begin{array}{|l|l|l|l|l|l|l|l|}
\hline 8 & 4 & 3 & 2 & 7 & 6 & 1 & 5 \\
\hline D&D &D &F &A &F &G &G \\
\hline A& F&G & V& D& A&A & X\\
\hline D& X&D & E& V& X&A & F\\
\hline D& X&G & V& A& F&V & X\\
\hline F& X&D & A& D& F&A & A\\
\hline G& V&F & D& D& X&V &X \\
\hline B& X& A&V &D &A &{\color{blue}X} &{\color{blue}F} \\
\hline
\end{array}
$$


填充过程中,因最后两个没有足够的明文让其填上,可规定字母$X$作为填充物,$X$在ADFGVX矩阵中的坐标是$XF$,故最后填充物为$XF$。

然后按照列表顺序从上到下依次读取密文。

$$
\begin{array}{|l|l|l|l|l|l|l|l|}
\hline 8 & 4 & 3 & 2 & 7 & 6 & 1 & 5 \\
\hline D&D &D &{\color{red}F} &A &F &{\color{blue}G} &G \\
\hline A& F&G & {\color{red}V}& D& A&{\color{blue}A} & X\\
\hline D& X&D & {\color{red}E}& V& X&{\color{blue}A} & F\\
\hline D& X&G & {\color{red}V}& A& F&{\color{blue}V} & X\\
\hline F& X&D & {\color{red}A}& D& F&{\color{blue}A} & A\\
\hline G& V&F & {\color{red}F}& D& X&{\color{blue}V} &X \\
\hline B& X& A&{\color{red}V} &D &A &{\color{blue}X} &F \\
\hline
\end{array}
$$

密文:{\color{blue}GAAVAVX} \quad {\color{red}FVGVADV} \quad DGDGDFA\quad DFXXXVX \quad  GXFXAXF \quad FAXFFXA \quad ADVADDD \quad DADDFGB 

\subsection{解密步骤}

与加密步骤反之。ADFGVX密码被法国人 Georges Painvin 破解。

\begin{enumerate}
\item 将密文的字母填入一个由n个长度组成的列表中,从上到下及从左到右逐列填写条目。
\item 按照规定列表重新排列。
\item 从左到右,从上到下,逐行读取。然后将结果文本分成每两个字母一组。
\item 使用ADFGVX表格中的坐标将每对转换为明文,坐标的顺序为(行索引、列索引)。
\end{enumerate}






































\end{document}

