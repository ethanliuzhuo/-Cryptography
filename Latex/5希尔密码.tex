\documentclass{article}
\usepackage[UTF8]{ctex}
\usepackage[utf8]{inputenc} % allow utf-8 input
\usepackage[T1]{fontenc}    % use 8-bit T1 fonts
\usepackage{hyperref}       % hyperlinks
\usepackage{url}            % simple URL typesetting
\usepackage{booktabs}       % professional-quality tables
\usepackage{amsfonts}       % blackboard math symbols
\usepackage{nicefrac}       % compact symbols for 1/2, etc.
\usepackage{microtype}      % microtypography
\usepackage{lipsum}
\usepackage{geometry}
\usepackage{graphicx} %插入图片的宏包
\usepackage{float} %设置图片浮动位置的宏包
\usepackage{subfigure} %插入多图时用子图显示的宏包
\usepackage{listings}

\geometry{a4paper,scale=0.8}
\date{}

\usepackage{listings}
\usepackage{color}

\definecolor{dkgreen}{rgb}{0,0.6,0}
\definecolor{gray}{rgb}{0.5,0.5,0.5}
\definecolor{mauve}{rgb}{0.58,0,0.82}

\lstset{frame=tb,
  language=Python,
  aboveskip=3mm,
  belowskip=3mm,
  showstringspaces=false,
  columns=flexible,
  basicstyle={\small\ttfamily},
  numbers=none,
  numberstyle=\tiny\color{gray},
  keywordstyle=\color{blue},
  commentstyle=\color{dkgreen},
  stringstyle=\color{mauve},
  breaklines=true,
  breakatwhitespace=true,
  tabsize=3
}

\title{希尔密码}


\author{
Hill Cipher\\
 刘卓\\
 \texttt{ } \\
}

\begin{document}
\maketitle

\section{矩阵的模运算}
矩阵的模运算(Modulo arithmetic on matrices)。令$A, B$ 是 $m \times n$ 的矩阵,矩阵元素都为整数。如果
$$
a_{i, j} \equiv b_{i, j} \quad \bmod m
$$

对于全部的 $a_{i, j}, b_{i, j}$. 则$A$ and $B$ 是模m的同余.记作$A \equiv B \bmod m$.

如果存在$A,B $是$ n \times n$矩阵,其元素都为整数,使得
$$AB = I \bmod m \mbox{和} BA = I \bmod m,$$那么$A^{-1} = B \bmod m$, B是A模m的逆。


\textbf{例1}

令$m = 5,$ $A=\left[\begin{array}{ll}2 & 3 \\ 2 & 1\end{array}\right]$ 和 $B=\left[\begin{array}{ll}1 & 2 \\ 3 & 2\end{array}\right] .$

\begin{eqnarray}   
\label{eq}
A+2 B \bmod 5&=& \left[\begin{array}{ll}
2 & 3 \\
2 & 1
\end{array}\right]+\left[\begin{array}{ll}
2 & 4 \\
{\color{red}6} & 4
\end{array}\right] \pmod{5} \nonumber \\ 
&=& \left[\begin{array}{ll}
2 & 3 \\
2 & 1
\end{array}\right]+\left[\begin{array}{ll}
2 & 4 \\
{\color{red}1} & 4
\end{array}\right] \pmod{5} \nonumber \\ 
&=& \left[\begin{array}{ll}
4 & 7 \\
3 & 5
\end{array}\right]  \pmod{5}  \nonumber \\ 
&=& \left[\begin{array}{ll}
4 & 2 \\
3 & 0
\end{array}\right]  \pmod{5}  \nonumber \\ 
\nonumber 
\end{eqnarray}


\begin{eqnarray}   
\label{eq}
BA \bmod 5&=& \left[\begin{array}{ll}
1 & 2 \\
3 & 2
\end{array}\right]\left[\begin{array}{ll}
2 & 3 \\
2 & 1
\end{array}\right] \pmod{5} \nonumber \\ 
&=& \left[\begin{array}{ll}
6 & 5 \\
10 & 11
\end{array}\right] \pmod{5} \nonumber \\ 
&=& \left[\begin{array}{ll}
1 & 0 \\
0 & 1
\end{array}\right]  \pmod{5} \nonumber \\ 
\nonumber 
\end{eqnarray}

\clearpage

\section{矩阵行列式的模}
矩阵行列式(determinant)。

$$det(A) \bmod m$$

\textbf{例2}

$$
A=\left[\begin{array}{cc}
3 & 4 \\
-9 & 8
\end{array}\right]  \bmod 10
$$

$$
\begin{aligned}
\operatorname{det}(A)=a d-b c=(3)(8)-(-9)(4) &=60 \\
=& 0 \bmod 10
\end{aligned}
$$

\section{希尔密码}
希尔密码(Hill cipher)是一种分组密码,将成对的明文
字母通过转换加密:
$$Y=AX \bmod 26$$其中A是一个$2 \times 2$的可逆矩阵(invertible matrix)并且mod 26.

模26的逆:

$$
\begin{array}{cc}
 {1^{-1} \equiv 1(\bmod 26)} & 3^{-1} \equiv 9(\bmod 26) \\
 9^{-1} \equiv 3(\bmod 26) & 5^{-1} \equiv 21(\bmod 26) \\ 
 21^{-1} \equiv 5(\bmod 26) & 7^{-1} \equiv 15(\bmod 26) \\ 
 15^{-1} \equiv 7(\bmod 26) & 11^{-1} \equiv 19(\bmod 26) \\
 19^{-1} \equiv 11(\bmod 26) & 17^{-1} \equiv 23(\bmod 26)\\
 23^{-1} \equiv 17(\bmod 26) & 25^{-1} \equiv 25(\bmod 26)\\
\end{array}
$$

~\\

\textbf{例3}

令$A=\left[\begin{array}{cc}
22 & 13 \\
11 & 5
\end{array}\right]$,使用希尔密码加密明文“$MISSING$”.

解:

\setlength{\tabcolsep}{1mm}{
\begin{tabular}{|c|c|c|c|c|c|c|c|c|c|c|c|c|c|c|c|c|c|c|c|c|c|c|c|c|c|}% 通过添加 | 来表示是否需要绘制竖线
\hline  % 在表格最上方绘制横线
0&1&2&3&4&5&6&7&8&9&10&11&12&13&14&15&16&17&18&19&20&21&22&23&24&25\\
\hline  %在第一行和第二行之间绘制横线
A&B&C&D&E&F&G&H&I&J&K&L&M&N&O&P&Q&R&S&T&U&V&W&X&Y&Z\\
\hline % 在表格最下方绘制横、
\end{tabular}
}

$$
\begin{array}{l}
{\left[\begin{array}{l}
M \\
I
\end{array}\right] \rightarrow\left[\begin{array}{l}
12 \\
8
\end{array}\right] \rightarrow AX \bmod 26 \rightarrow \left[\begin{array}{cc}
22 & 13 \\
11 & 5
\end{array}\right]\left[\begin{array}{l}
12 \\
8
\end{array}\right]} \bmod 26  \rightarrow \left[\begin{array}{l}
368 \\
172
\end{array}\right] \bmod 26 \rightarrow \left[\begin{array}{l}
4 \\
16
\end{array}\right] \rightarrow\left[\begin{array}{l}
E \\
Q
\end{array}\right]\\
{\left[\begin{array}{l}
S \\
S
\end{array}\right] \rightarrow\left[\begin{array}{l}
18 \\
18
\end{array}\right] \rightarrow\left[\begin{array}{l}
6 \\
2
\end{array}\right] \rightarrow\left[\begin{array}{l}
\mathrm{G} \\
\mathrm{C}
\end{array}\right]} \\
{\left[\begin{array}{l}
I \\
N
\end{array}\right] \rightarrow\left[\begin{array}{l}
8 \\
13
\end{array}\right] \rightarrow \left[\begin{array}{l}
7 \\
23
\end{array}\right] \rightarrow\left[\begin{array}{l}
H \\
X
\end{array}\right]} \\
{\left[\begin{array}{l}
G \\
{\color{red}K}
\end{array}\right] \rightarrow\left[\begin{array}{l}
6 \\
{\color{red}10}
\end{array}\right] \rightarrow\left[\begin{array}{l}
2 \\
{\color{red}12}
\end{array}\right] \rightarrow\left[\begin{array}{l}
C \\
{\color{red}M}
\end{array}\right]}
\end{array}
$$

遇到明文长度为奇数时,可以规定某个字母为填充项。这里使用字母${\color{red}K}$作为填充项。

最后密文为:$EQGCHXCM$ \quad\quad\quad\quad\quad\quad\quad\quad\quad\quad\quad\quad\quad\quad\quad\quad\quad\quad\quad\quad\quad\quad\quad\quad\quad\quad\quad   \Box

~\\

希尔密码解密为:$X = A^{-1}Y \bmod 26$

\textbf{例4}

解密"$ZGWQ$",令$A=\left[\begin{array}{cc}
3 & 7 \\
9 & 10
\end{array}\right]$

解:

\begin{eqnarray}   
\label{eq}
A^{-1}&=&\frac{1}{det(A)}\left[\begin{array}{cc}
d & -b \\
-c & a
\end{array}\right] \bmod 26 \nonumber \\ 
&=& \operatorname{det}(A)^{-1} \cdot\left[\begin{array}{cc}
10 & -7 \\
-9 & 3
\end{array}\right] \bmod 26 \nonumber \\ 
&=&\operatorname{det(A)}^{-1}\left[\begin{array}{cc}
10 & 19 \\
17 & 3
\end{array}\right]  \bmod 26 \nonumber \\ 
&=& (-33\bmod 26)^{-1} \times \left[\begin{array}{cc}
10 & 19 \\
17 & 3
\end{array}\right]  \bmod 26 \nonumber \\ 
&=& 19^{-1} \times \left[\begin{array}{cc}
10 & 19 \\
17 & 3
\end{array}\right]  \bmod 26 \nonumber \\ 
&=& 11 \times \left[\begin{array}{cc}
10 & 19 \\
17 & 3
\end{array}\right]  \bmod 26 \nonumber \\ 
&=&\left[\begin{array}{cc}
6 & 1 \\
5 & 7
\end{array}\right]  \bmod 26 \nonumber \\ 
\nonumber 
\end{eqnarray}

$$
\begin{array}{l}
{\left[\begin{array}{l}
Z \\
G
\end{array}\right] \rightarrow\left[\begin{array}{c}
25 \\
6
\end{array}\right] \rightarrow\left[\begin{array}{ll}
6 & 1 \\
5 & 7
\end{array}\right]\left[\begin{array}{c}
25 \\
6
\end{array}\right] \bmod 26=\left[\begin{array}{l}
0 \\
11
\end{array}\right] \rightarrow\left[\begin{array}{l}
A \\
L
\end{array}\right]} \\
\left[\begin{array}{l}
W \\
Q
\end{array}\right] \rightarrow\left[\begin{array}{l}
22 \\
16
\end{array}\right] \rightarrow\left[\begin{array}{ll}
6&1 \\
5&7
\end{array}\right]\left[\begin{array}{l}
22 \\
16
\end{array}\right] \bmod 26 =\left[\begin{array}{l}
18 \\
14
\end{array}\right] \rightarrow\left[\begin{array}{l}
S \\
O
\end{array}\right]
\end{array}
$$

解密字符为:"ALSO"

\clearpage

\textbf{例5}

加密CODE,令$A=\left[\begin{array}{cc}
22 & 13 \\
11 & 5
\end{array}\right]$
\begin{lstlisting}
import numpy as np

Plaintext = 'COKE'#字母需要大写

A  = np.mat([[22,13],[11,5]])

Ciphertext = ''

if len(Plaintext)%2 == 0:
    pass
else:
    Plaintext += 'K' #填充K

for i in range(len(Plaintext)):
    if i%2 == 0:
        M = Plaintext[i:i+2]
        
        X = np.mat([[ord(M[0]) -65],[ord(M[1]) -65]]) 
        Y = A*X%26
        
        Ciphertext += chr(int(Y[0]) + 65)
        Ciphertext += chr(int(Y[1]) + 65)
print(Ciphertext)    
\end{lstlisting}

输出:SOMA

\clearpage

\textbf{例6}

已知密文"$DLHIVDLZHIPNEU$",已知使用希尔密码和$2 \times 2$作为加密手段。并且有明显证据证明明文开头为$DEAR$。尝试解密。

解:

\setlength{\tabcolsep}{2mm}{
\begin{tabular}{|c|c|c|c|c|c|c|c|c|c|c|c|c|c|}% 通过添加 | 来表示是否需要绘制竖线
\hline  % 在表格最上方绘制横线
{\color{red}D}&{\color{red}L}&{\color{red}H}&{\color{red}I}&V&D&L&Z&H&I&P&N&E&U \\\hline  %在第一行和第二行之间绘制横线
3&11&7&8&21&3&11&25&7&8&15&13&4&20 \\\hline % 在表格最下方绘制横、
\end{tabular}
}

\setlength{\tabcolsep}{2mm}{
\begin{tabular}{|c|c|c|c|}% 通过添加 | 来表示是否需要绘制竖线
\hline  % 在表格最上方绘制横线
{\color{blue}D}&{\color{blue}E}&{\color{blue}A}&{\color{blue}R} \\\hline  %在第一行和第二行之间绘制横线
3&4&0&17 \\\hline % 在表格最下方绘制横、
\end{tabular}
}


$$
\left[\begin{array}{l}
D \\
L
\end{array}\right] \mapsto\left[\begin{array}{l}
D \\
E
\end{array}\right]
$$

$$
\left[\begin{array}{l}
H \\
I
\end{array}\right] \mapsto\left[\begin{array}{l}
V \\
D
\end{array}\right]
$$

下一步计算$A^{-1}$,

$$
\begin{array}{l}
A^{-1}\left[\begin{array}{l}
3 \\
11
\end{array}\right]=\left[\begin{array}{l}
3 \\
4
\end{array}\right] \bmod 26 \\
A^{-1}\left[\begin{array}{l}
7 \\
8
\end{array}\right]=\left[\begin{array}{l}
0 \\
17
\end{array}\right] \bmod 26
\end{array}
$$

因为:
$$
\begin{aligned}
A^{-1}\left[\begin{array}{ll}
3 & 7 \\
11 & 8
\end{array}\right] = \left[\begin{array}{ll}
3 & 0 \\
4 & 17
\end{array}\right]

所以:
$$
\begin{aligned}
A^{-1} &{=\left[\begin{array}{ll}
3 & 0 \\
4 & 17
\end{array}\right] \cdot\left[\begin{array}{ll}
3 & 7 \\
11 & 8
\end{array}\right]^{-1}} = {\left[\begin{array}{ll}
3 & 0 \\
4 & 17
\end{array}\right] \cdot\left[\begin{array}{ll}
18 & 7 \\
11 & 23
\end{array}\right]=\left[\begin{array}{ll}
2 & 21 \\
25 & 3
\end{array}\right] \bmod 26}
\end{aligned}
$$

$X = A^{-1} \cdot Y$,与例4同理



























































\end{document}

