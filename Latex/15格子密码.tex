\documentclass{article}
\usepackage[UTF8]{ctex}
\usepackage[utf8]{inputenc} % allow utf-8 input
\usepackage[T1]{fontenc}    % use 8-bit T1 fonts
\usepackage{hyperref}       % hyperlinks
\usepackage{url}            % simple URL typesetting
\usepackage{booktabs}       % professional-quality tables
\usepackage{amsfonts}       % blackboard math symbols
\usepackage{nicefrac}       % compact symbols for 1/2, etc.
\usepackage{microtype}      % microtypography
\usepackage{lipsum}
\usepackage{amsmath}
\usepackage{geometry}
\usepackage{graphicx} %插入图片的宏包
\usepackage{float} %设置图片浮动位置的宏包
\usepackage{subfigure} %插入多图时用子图显示的宏包
\usepackage{listings}
\usepackage{amsthm}
\usepackage{tikz}

\theoremstyle{definition}
\newtheorem{theorem}{\indent 定理}[section]
\newtheorem{lemma}[theorem]{\indent 引理}
\newtheorem{proposition}[theorem]{\indent 命题}
\newtheorem{corollary}[theorem]{\indent 推论}
\newtheorem{definition}{\indent 定义}
\newtheorem{example}{\indent 例}
\newtheorem{remark}{\indent 注}
\newenvironment{solution}{\begin{proof}[\indent\bf 解]}{\end{proof}}
\renewcommand{\proofname}{\indent\bf 证明}

\usepackage[utf8]{inputenc}
\usepackage[T1]{fontenc}
\usepackage{pgfplots}
\usetikzlibrary{calc}
\pgfplotsset{compat=1.12}
    
\pgfplotsset{
    compat=1.12,
}

\geometry{a4paper,scale=0.8}
\date{}

\usepackage{listings}
\usepackage{color}

\definecolor{dkgreen}{rgb}{0,0.6,0}
\definecolor{gray}{rgb}{0.5,0.5,0.5}
\definecolor{mauve}{rgb}{0.58,0,0.82}

\lstset{frame=tb,
  language=Python,
  aboveskip=3mm,
  belowskip=3mm,
  showstringspaces=false,
  columns=flexible,
  basicstyle={\small\ttfamily},
  numbers=none,
  numberstyle=\tiny\color{gray},
  keywordstyle=\color{blue},
  commentstyle=\color{dkgreen},
  stringstyle=\color{mauve},
  breaklines=true,
  breakatwhitespace=true,
  tabsize=3
}

\title{格子密码}


\author{
Lattice-based Cryptography\\
 刘卓\\
 \texttt{ } \\
}


\begin{document}
\maketitle

格子密码是在结构本身或在安全证明中涉及格(Lattice)的加密图元构造的通用术语。基于晶格的构造目前是后量子密码学的重要候选者。

\section{格}

\begin{definition}
让$\mathbf{v_1,v_2,\cdots,v_n}$是在$n$维欧几里得空间(Euclidean space)$R^n$中的独立向量(independent vectors)。格(lattice)记作$L$,它是由$\{\mathbf{v_1,v_2,\cdots,v_n}\}$的线性组合的集合,$\mathbf{v_1,v_2,\cdots,v_n}$的系数为整数:
$$L = \{a_1\mathbf{v_1}+a_2\mathbf{v_2}+\cdots +a_n\mathbf{v_n},\mathbf{a_i} \in \mathbb{Z}\}$$
\end{definition}

备注:$L$的基是生成$L$的n个独立向量的任意集合。格子有无穷多个基。

\begin{definition}
设$L$是维数为$n$的格,\mathbb{B} = $\{\mathbf{v_1,v_2,\cdots,v_n}\}$是$L$的一个基。$L$的基本域(或基本平行六面体)对应于这个基$B$是集合:
$$\mathbb{F}(\mathbf{v_1,v_2,\cdots,v_n}) = \{t_1\mathbf{v_1}+t_2\mathbf{v_2}+\cdots +t_n\mathbf{v_n}, 0 \leq t_i  < 1\}$$
\end{definition}

\begin{lemma}
设$L \subset \mathbb{R^n} $是$n$维的格,\mathbb{F}是$L$的一个基本域。则每一个向量$\mathbf{w} \in \mathbb{R^n}$可以写作:
$$\mathbf{w = t + v}$$
有唯一的$\mathbf{t} \in \mathbb{F}$和有唯一的$\mathbf{v} \in L$。
\end{lemma}

$$
    \begin{tikzpicture}[
        % define the style `point' which is used for the nodes on the coordinates
        point/.style={
            circle,
            fill=blue,
            inner sep=1.5pt,
        },
    ]
        \begin{axis}[
            xmin=-4,
            xmax=7,
            ymin=-7,
            ymax=7,
            xlabel={$x$},
            ylabel={$y$},
            scale only axis,
            axis lines=middle,
            domain=-1.912931:3,
            samples=200,
            smooth,
            clip=false,
            % use same unit vectors on the axis
            axis equal image=true,
        ]

            % add nodes to the points and the corresponding labels
            \node [point,label={{\color{red}}}]
                (P0) at (0,0)  {{\color{red}}};
                
            \node [point,label={}]
                (P1) at (-4,-1)  {};
            \node [point,label={}]
                (P2) at (-2,-1)  {};
            \node [point,label={}]
                (P3) at (0,-1)  {};
             \node [point,label={}]
                (P4) at (2,-1)  {};
             \node [point,label={}]
                (P5) at (4,-1)  {};
            \node [point,label={}]
                (P6) at (-3,0)  {};
            \node [point,label={}]
                (P7) at (-1,0) {};
            \node [point,label={}]
                (P8) at (1,0)  {};
            \node [point,label={}]
                (P30) at (3,0)  {};
            \node [point,label={}]
                (P10) at (5,0)  {};
            \node [point,label={}]
                (P11) at (-2,1) {};
            \node [point,label={}]
                (P12) at (0,1)  {};
             \node [point,label={}]
                (P21) at (2,1)  {};
             \node [point,label={}]
                (P14) at (4,1)  {};
            \node [point,label={}]
                (P15) at (6,1) {};
            
            \node [point,label={}]
                (P7) at (-1,2) {};
            \node [point,label={}]
                (P8) at (1,2)  {};
            \node [point,label={}]
                (P32) at (3,2)  {};
            \node [point,label={}]
                (P10) at (5,2)  {};
            \node [point,label={}]
                (P6) at (7,2)  {};
                
           
            \node [point,label={}]
                (P12) at (0,3)  {};
             \node [point,label={}]
                (P13) at (2,3)  {};
             \node [point,label={}]
                (P14) at (4,3)  {};
            \node [point,label={}]
                (P15) at (6,3) {};
             \node [point,label={}]
                (P11) at (8,3) {};


            % draw a line from (P0) a bit further than just to (P3)
            \draw [red,->] (P0) -- node[red,below]  {$v_1$} (P30);
            \draw [red,->] (P0) -- node[red,above]  {$v_2$} (P32);
        \end{axis}
    \end{tikzpicture}
$$


\textbf{寻找格规约算法}(An algorithm for finding reduced basis):
\begin{enumerate}
\item while $\left\|\mathbf{b}_{\mathbf{1}}\right\|>\left\|\mathbf{b}_{\mathbf{2}}\right\|$
\item  \quad\quad\quad $\operatorname{swap} \mathbf{b}_{1}, \mathbf{b}_{2}$
\item \quad\quad\quad$u:=\left[\frac{b_{1} \cdot b_{2}}{b_{1} \cdot b_{1}}\right]$($[u]$取最接近的整数)
\item \quad\quad\quad if $u=0$ then return $\left(\mathbf{b}_{1}, \mathbf{b}_{2}\right)$ 
\item \quad\quad\quad else $\mathbf{b}_{2}:=\mathbf{b}_{2}-u \cdot \mathbf{b}_{1}$
\end{enumerate}

~\\

\begin{example}
$\mathbf{b}_{1} = (90,123), \mathbf{b}_{2} = (56,76)$

寻找格规约(reduced basis)

\begin{solution}

\textbf{第一步:}
$$
\begin{array}{l}
\left\|b_{1}\right\| =  3 \sqrt{2581} \\
\left\|b_{2}\right\|=4 \sqrt{557}
\end{array} 
\left\|b_{1}\right\|>\left\|b_{2}\right\|
$$

交换$\mathbf{b}_{1},\mathbf{b}_{2}$, $\mathbf{b}_{1} = (56,76), \mathbf{b}_{2} = (90,123)$

$$u:=\left[\frac{b_{1} \cdot b_{2}}{b_{1} \cdot b_{1}}\right] = \left[\frac{56 \cdot 90 + 76 \cdot 123}{56^2 + 76^2}\right] = \left[\frac{3597}{2228}\right] = 2 \ne 0$$

$$
\mathbf{b}_{2}=(90,123)-2(56,-76)=(-22,-29)
$$

\textbf{第二步:}

$\mathbf{b}_{1} = (56,76) , \mathbf{b}_{2} =(-22,-29)$

$$
\begin{array}{l}
\left\|b_{1}\right\|= 4 \sqrt{557}\\
\left\|b_{2}\right\|=5 \sqrt{53}
\end{array} 
\left\|b_{1}\right\|>\left\|b_{2}\right\|
$$

交换$\mathbf{b}_{1},\mathbf{b}_{2}$,$\mathbf{b}_{1} =  (-22,-29), \mathbf{b}_{2} = (56,76)$

$$u:=\left[\frac{b_{1} \cdot b_{2}}{b_{1} \cdot b_{1}}\right] = \left[\frac{(-22) \cdot 56 + (-29) \cdot 76}{(-22)^2 + (-29)^2}\right] = \left[-\frac{3436}{1325}\right] = -3 \ne 0$$

$$
\mathbf{b}_{2}=\mathbf{b}_{2} + 3\mathbf{b}_{1} = (-10,-11)
$$

\textbf{第三步:}

$\mathbf{b}_{1} = (-22,-29), \mathbf{b}_{2} =(-10,-11)$

$$
\begin{array}{l}
\left\|b_{1}\right\|= 5 \sqrt{53}\\
\left\|b_{2}\right\|= \sqrt{221}
\end{array} 
\left\|b_{1}\right\|>\left\|b_{2}\right\|
$$

交换$\mathbf{b}_{1},\mathbf{b}_{2}$,$\mathbf{b}_{1} =  (-10,-11), \mathbf{b}_{2} = (-22,-29)$

$$u:=\left[\frac{b_{1} \cdot b_{2}}{b_{1} \cdot b_{1}}\right] = \left[\frac{539}{221}\right] = 2 \ne 0$$

$$
\mathbf{b}_{2}=\mathbf{b}_{2} - 2\mathbf{b}_{1} = (-2,-7)
$$

\textbf{第四步:}

$\mathbf{b}_{1} = (-10,-11), \mathbf{b}_{2} =(-2,-7)$

$$
\begin{array}{l}
\left\|b_{1}\right\|= \sqrt{221}\\
\left\|b_{2}\right\|= \sqrt{53}
\end{array} 
\left\|b_{1}\right\|>\left\|b_{2}\right\|
$$

交换$\mathbf{b}_{1},\mathbf{b}_{2}$,$\mathbf{b}_{1} =  (-2,-7), \mathbf{b}_{2} = (-10,-11)$

$$u:=\left[\frac{b_{1} \cdot b_{2}}{b_{1} \cdot b_{1}}\right] = \left[-\frac{97}{53}\right] = 1.8 = 2 \ne 0$$

$$
\mathbf{b}_{2}=\mathbf{b}_{2} - 2\mathbf{b}_{1} = (-6,3)
$$

\textbf{第五步:}

$\mathbf{b}_{1} = (-2,-7), \mathbf{b}_{2} =(-6,3)$

$$
\begin{array}{l}
\left\|b_{1}\right\|= \sqrt{53}\\
\left\|b_{2}\right\|= \sqrt{39}
\end{array} 
\left\|b_{1}\right\|>\left\|b_{2}\right\|
$$

交换$\mathbf{b}_{1},\mathbf{b}_{2}$,$\mathbf{b}_{1} =  (-6,3), \mathbf{b}_{2} = (-2,-7)$

$$u:=\left[\frac{b_{1} \cdot b_{2}}{b_{1} \cdot b_{1}}\right] = \left[\frac{-9}{39}\right]  =  0$$


结束,返回$\mathbf{b}_{1} =  (-6,3), \mathbf{b}_{2} = (-2,-7)$


\end{solution}
\end{example}



\section{最短向量问题}
最短向量问题(The Shortest Vector Problem,SVP)是指在格$L$中找到一个最短的非零向量,即找到一个非零向量 $v \in L$,使欧几里得范数$||\mathbf{v}|| = \sqrt{v_1^2 + v_2^2 + \cdots + v_n^2}$最小。

最近向量问题(The Closest Vector Problem ,CVP)是一个向量$\mathbf{w \in \mathbb{R}}$ 不在$L$中,然后找到一个向量$\mathbf{v \in L}$ 与$\mathbf{w}$最接近。他们之间的欧氏距离范数为$||\mathbf{w-v}||$。

SVP和CVP都是深奥的问题,随着格的维度的增加,这两个问题的计算都变得越来越困难。另一方面,SVP和CVP的近似解在纯数学和应用数学的不同领域都有令人惊讶的应用。

如果维数n很大,例如$n > 100 $ 那么已知的算法在求解SVP时并不是很有效,但$CVP$问题却很容易。这允许在密码结构中使用格,几种基于格的密码系统已经被提出。

\section{Goldreich, Goldwasser和Halevi (GGH)公钥密码系统}
\begin{enumerate}
\item ${\color{blue}Alice}$ 在整数域选择一个基(basis) $\mathbf{v = \{v_1,v_2,\cdots,v_n\}}$。设$\mathbf{V}$是矩阵它的行是向量$v_i$,$L$是向量$v_i$生成的格;
\item ${\color{blue}Alice}$ 选择一个$n \times n$的矩阵$U$,$U$都是整数项,并且$det(U) = \pm 1$。计算$W = UV$。然后公开$W$的行向量$\mathbf{w_i}$;
\item 假设${\color{red}Bob}$想要发送信息$\mathbf{m} = (m_1,_2,\cdots,m_n)$,将$\mathbf{m}$转化为一个整数域的向量。然后随机选择一个小向量$\mathbf{r}$,计算$$\mathbf{e} = m_1\mathbf{w_1} +  m_2\mathbf{w_2} + \cdots +  m_n\mathbf{w_n} + \mathbf{r}$$
将$ \mathbf{e}$发送给${\color{blue}Alice}$ ;
\item ${\color{blue}Alice}$ 解决最近向量问题(CVP),找到一个向量$\mathbf{d} \in L$与$\mathbf{e}$最接近。然后计算

$$
\mathbf{d} \cdot W^{-1}=\left(m \cdot W+r\right) W^{-1}=m \cdot U \cdot W \cdot W^{-1}+r \cdot W^{-1}=m \cdot U+r \cdot W^{-1}
$$

$\mathbf{m} = \mathbf{m} \cdot U \cdot U^{-1}$;
\end{enumerate}

~\\

\begin{example}
令$$
V= \begin{pmatrix}
11 & 0 \\
0 & 7
\end{pmatrix}
 \quad\quad
U=\begin{pmatrix}
7 & 5 \\
4 & 3
\end{pmatrix}
$$

使用$\mathbf{r} = (1,1)$加密信息$\mathbf{m} = (12,13)$

\begin{solution}

验证$det(U) = 7 \cdot 3 - 4 \cdot 5 = 1$满足条件,可以使用

$$
V^{-1}=\begin{pmatrix}
\frac{1}{11} & 0 \\
0 & \frac{1}{7}
\end{pmatrix}
\quad\quad
U^{-1}=\begin{pmatrix}
3 & -5 \\
-4 & 7
\end{pmatrix}
$$

$$W = UV = \begin{pmatrix}
7 & 5 \\
4 & 3
\end{pmatrix}\begin{pmatrix}
11 & 0 \\
0 & 7
\end{pmatrix} = \begin{pmatrix}
77 & 35 \\
44 & 21
\end{pmatrix}  
$$

$$\mathbf{e} = \mathbf{m}W + \mathbf{r} = (1497       ,694)$$

${\color{blue}Alice}$收到$\mathbf{e}$后进行解密:

$$\mathbf{e} V^{-1} = (1497,694) \begin{pmatrix}
\frac{1}{11} & 0 \\
0 & \frac{1}{7}
\end{pmatrix} = \left(
\frac{1497}{11},  \frac{694}{7} \right) \approx (136,99)$$


$$ \mathbf{m} = (136,99)U^{-1} =  (136,99)\begin{pmatrix}
3 & -5 \\
-4 & 7
\end{pmatrix}= (12,13)$$



\end{solution}

\end{example}



~\\
$\mathbf{NTRU}$公钥密码系统是由Hoffstein, Pipher和 Silverman发明。由一种多项式的卷积所构成。

多项式卷积算法是$$f * g = \sum_{k=0}^{N-1}c_kx^k$$ 其中 $$c_k = \sum_{i+j \equiv k \pmod{N}} f_ig$$

~\\

\begin{example}
计算当$N = 3$时,$f * g$,其中 

$$
f= \sum_{i=0}^{N-1} f_{i} x^{i}=x^{2}+4 x-1 
$$
$$
g =\sum_{i=0}^{N-1} g_{i} x^{i}=5 x^{2}-x+3 
$$

~\\

\begin{solution}

$f$的多项式为$f = [f_0,f_1,f_2] = [-1,4,1]$

$g$的多项式为$g = [g_0,g_1,g_2] = [3,-1,5]$

$c = f * g$的多项式是

$$
\left[\begin{array}{c}
c_{0} \\
c_{1} \\
c_{2}
\end{array}\right]=\left[\begin{array}{ccc}
f_{0} & f_{2} & f_{1} \\
f_{1} & f_{0} & f_{2} \\
f_{2} & f_{1} & f_{0}
\end{array}\right]\left[\begin{array}{c}
g_{0} \\
g_{1} \\
g_{2}
\end{array}\right]=\left[\begin{array}{ccc}
-1 & 1 & 4 \\
4 & -1 & 1 \\
1 & 4 & -1
\end{array}\right]\left[\begin{array}{c}
3 \\
-1 \\
5
\end{array}\right]=\left[\begin{array}{c}
16 \\
18 \\
-6
\end{array}\right]
$$

$$f * g = 16+18x -6x^2$$

因此通常情况$f * g$通项是:

$$
\left[\begin{array}{c}
c_{0} \\
c_{1} \\
\vdots \\
c_{N-1}
\end{array}\right]=\left[\begin{array}{ccccc}
f_{0} & f_{N-1} & f_{N-2}  &  \cdots &   f_{1}   \\
f_{1} & f_{0} & f_{N-1}  &  \cdots &   f_{2}  \\
\vdots & \vdots& \vdots  &  \ddots &   \vdots  \\
f_{N-1} & f_{N-2} & f_{N-3} &  \cdots &  f_{0} \\ 
\end{array}\right]\left[\begin{array}{c}
g_{0} \\
g_{1} \\
\vdots \\
g_{N-1}
\end{array}\right]=\left[\begin{array}{c}
p_{0} \\
p_{1} \\
\vdots \\
p_{N-1}
\end{array}\right]
$$
\end{solution}
\end{example}

\subsection{NTRU}
NTRU是一种非对称密码系统,它基于格上的最短向量问题。1996年由Hoffstein, piphher和Silverman发现。2009年,NTRU密码系统已经通过了电气和电子工程师协会(IEEE)的标准化认证。与RSA和ECC相比,这是一种更有效的加密和解密方法:更快的密钥生成(使用一次性密钥)和低内存使用(因此它可以用于移动设备和智能卡)。

NTRU的消息为多项式形式,其系数为整数mod $p$。

~\\

${\color{blue}Alice}$作为发送者以及${\color{red}Bob}$接受者想使用NTRU,那么他们该如何使用呢? 
\begin{enumerate}
\item ${\color{blue}Alice}$和${\color{red}Bob}$共同决定一个指数$N$,使得多项式的次数最多为$N-1$,系数为整数。以及再共同决定两个整数$p,q$,使得$gcd(p,q) = 1$;
\item ${\color{red}Bob}$选择两个多项式
$$f = \sum_{k = 0}^{N-1}f_ix^i$$
$$g = \sum_{k = 0}^{N-1}g_ix^i$$
然后计算 $f_p := f^{-1} \pmod{ p}$和 $f_q := f^{-1} \pmod{ q}$,保留$f_p ,f_q$;
\item ${\color{red}Bob}$计算$h = f_q * g \pmod{q}$,并公开;
\item ${\color{blue}Alice}$把信息转换成一个多项式$m$,其系数是整数介于$-p/2$和$p/2$ mod $p$之间;
\item 然后${\color{blue}Alice}$随机选择另一个小的多项式$r$(以掩盖信息)。保留$r$不公开,然后将$e = pr * h + m \pmod{q}$发送给${\color{red}Bob}$;
\item ${\color{red}Bob}$ 为了还原信息,他需要计算$a = f * e \pmod{q}$和$b = a \pmod{p}$;
\item 最后,${\color{red}Bob}$计算$f_p * b \pmod{p}$还原信息;
\end{enumerate}

注意$f *f_p = 1 \pmod{p},f *f_q = 1 \pmod{q}, h = f_q * g \pmod{q}$

$$a = f * e = f * (pr * h + m) = pr * (f * h) + f*m = pr*g + f*m \pmod{q}$$

$$b = a \pmod{p} = pr * g + f * m \pmod{p} = f * m \pmod{p} $$
因此,
$$f_p * b = f_p *f * m = m \pmod{p} $$
还原成功。


\begin{example}密钥生成
${\color{blue}Alice}$和${\color{red}Bob}$ 共同决定了$N = 7,p = 3, q = 41,d = 2$。
\begin{enumerate}
\item ${\color{red}Bob}$选择了两个多项式:
$$f(x) = x^6−x^4+x^3+x^2−1 $$
$$g(x) = x^6+x^4−x^2−x$$
\item $$Fq(x) = f (x)^ {−1} \pmod{q} = 8x^6 + 26x^5 + 31x^4
+ 21x^3 + 40x^2 + 2x + 37 \pmod{41}$$
$$Fp(x) = f (x)^ −1 \pmod{p} = x^6 + 2x^5 + x^3 + x^2 +
x + 1  \pmod{3}$$ 保留为私钥;
\item $$h(x) = p * (Fq )*g \pmod{q} = 20x^6 + 40x^5 + 2x^4
+ 38x^3 + 8x^2 + 26x + 30 \pmod{41} $$ 并公开;
\item 假设${\color{blue}Alice}$ 把信息转化为 $m(x) = −x^5 +x^3 +x^2 − x + 1$ 并用$r(x) x^6 − x^5 + x −
1$进行加密
$$e(x)≡ 31x^6+19x^5+4x^4+2x^3+40x^2+3x+25 \pmod{41}$$
\item ${\color{red}Bob}$计算$$a = f * e \pmod{q} =  x^6 + 10x^5 + 33x^4 + 40x^3 + 40x^2 + x + 40 \pmod{41}$$
$$b = a \pmod{p} =x^6 + 10x^5 −8x^4 − x^3 − x^2 + x − 1  \pmod{3} $$
\item ${\color{red}Bob}$计算$f_p * b \pmod{p} = 2x^5 + x^3 + x^2 + 2x + 1 \pmod{3}  = {\color{red}m(x)} =−x^5 +x^3 +x^2 − x + 1$ ,还原成功
\end{enumerate}

\end{example}

$\hfill\square$ 

~\\

当$N = 503,p = 3, q = 256$,就拥有非常高的安全性了。

\section{Zero Knowledge Protocol}

假设有一个双面隧道,隧道两边有一扇门。
${\color{blue}Alice}$想向${\color{red}Bob}$证明她能开门。然而,她不想让${\color{red}Bob}$知道她是怎么做到的。该如何做呢?

\begin{enumerate}
\item ${\color{red}Bob}$秘密地随机选择k个命令序列,“对”、“错”、“左”、“右”之类的;
\item ${\color{blue}Alice}$进入隧道(${\color{red}Bob}$没有看到她从哪边进入),并站在门中等待${\color{red}Bob}$的命令;
\item ${\color{red}Bob}$发出命令,并观察${\color{blue}Alice}$离开隧道的一侧;
\item 如果${\color{blue}Alice}$每次都能从正确的一边出来,那么${\color{red}Bob}$就接受她可以打开门。
\end{enumerate}

\end{document}

%https://en.wikipedia.org/wiki/Lattice-based_cryptography
%http://people.scs.carleton.ca/~maheshwa/courses/4109/Seminar11/NTRU_presentation.pdf
%http://www.math.ucsd.edu/~qtbach/187/Day%2023A.pdf