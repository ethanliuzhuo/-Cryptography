\documentclass{article}
\usepackage[UTF8]{ctex}
\usepackage[utf8]{inputenc} % allow utf-8 input
\usepackage[T1]{fontenc}    % use 8-bit T1 fonts
\usepackage{hyperref}       % hyperlinks
\usepackage{url}            % simple URL typesetting
\usepackage{booktabs}       % professional-quality tables
\usepackage{amsfonts}       % blackboard math symbols
\usepackage{nicefrac}       % compact symbols for 1/2, etc.
\usepackage{microtype}      % microtypography
\usepackage{lipsum}
\usepackage{geometry}
\usepackage{graphicx} %插入图片的宏包
\usepackage{float} %设置图片浮动位置的宏包
\usepackage{subfigure} %插入多图时用子图显示的宏包
\usepackage{listings}
\usepackage{amsthm}

\theoremstyle{definition}
\newtheorem{theorem}{\indent 定理}[section]
\newtheorem{lemma}[theorem]{\indent 引理}
\newtheorem{proposition}[theorem]{\indent 命题}
\newtheorem{corollary}[theorem]{\indent 推论}
\newtheorem{definition}{\indent 定义}[section]
\newtheorem{example}{\indent 例}
\newtheorem{remark}{\indent 注}[section]
\newenvironment{solution}{\begin{proof}[\indent\bf 解]}{\end{proof}}
\renewcommand{\proofname}{\indent\bf 证明}


\geometry{a4paper,scale=0.8}
\date{}

\usepackage{listings}
\usepackage{color}

\definecolor{dkgreen}{rgb}{0,0.6,0}
\definecolor{gray}{rgb}{0.5,0.5,0.5}
\definecolor{mauve}{rgb}{0.58,0,0.82}

\lstset{frame=tb,
  language=Python,
  aboveskip=3mm,
  belowskip=3mm,
  showstringspaces=false,
  columns=flexible,
  basicstyle={\small\ttfamily},
  numbers=none,
  numberstyle=\tiny\color{gray},
  keywordstyle=\color{blue},
  commentstyle=\color{dkgreen},
  stringstyle=\color{mauve},
  breaklines=true,
  breakatwhitespace=true,
  tabsize=3
}

\title{凯撒密码}


\author{
The Caesar Cipher\\
 刘卓\\
 \texttt{ } \\
}

\begin{document}
\maketitle

凯撒密码,或称凯撒加密、凯撒变换、变换加密,是一种最简单且最广为人知的加密技术。距今已有2000余年的历史。

凯撒密码属于密码学中的替换加密,即密文是由明文中的所有字母在字母表上向后(或向前)按照一个固定数目进行偏移而生成。

%\begin{figure}[H] %H为当前位置,!htb为忽略美学标准,htbp为浮动图形
%\centering %图片居中
%\includegraphics[width=0.7\textwidth]{图库/1.1.jpg} %插入图片,[]中设置图片大小,{}中是图片文件名
%\caption{Main name 2} %最终文档中希望显示的图片标题
%\label{Fig.main2} %用于文内引用的标签
%\end{figure}

\section{加密}

首先需要设置偏移量:

\begin{example}

当偏移量为16时,得到如下加密方法。所有的字母\textbf{A}向拉丁字母表右移动16位,将被替换成字母\textbf{Q},字母\textbf{B}向字母表右移动16位变成字母\textbf{R},以此列推。

$$
\begin{array}{|l|l|l|l|l|l|l|l|l|l|l|l|l|l|}
\hline \mathrm{A} & \mathrm{B} & \mathrm{C} & \mathrm{D} & \mathrm{E} & \mathrm{F} & \mathrm{G} & \mathrm{H} & \mathrm{I} & \mathrm{J} & \mathrm{K} & \mathrm{L} & \mathrm{M} & \mathrm{N} \\
\hline \mathrm{Q} & \mathrm{R} & \mathrm{S} & \mathrm{T} & \mathrm{U} & \mathrm{V} & \mathrm{W} & \mathrm{X} & \mathrm{Y} & \mathrm{Z} & \mathrm{A} & \mathrm{B} & \mathrm{C} & \mathrm{D} \\
\hline
\end{array}\\
\begin{array}{|c|c|c|c|c|c|c|c|c|c|c|c|}
\hline \mathrm{O} & \mathrm{P} & \mathrm{Q} & \mathrm{R} & \mathrm{S} & \mathrm{T} & \mathrm{U} & \mathrm{V} & \mathrm{W} & \mathrm{X} & \mathrm{Y} & \mathrm{Z} \\
\hline \mathrm{E} & \mathrm{F} & \mathrm{G} & \mathrm{H} & \mathrm{I} & \mathrm{J} & \mathrm{K} & \mathrm{L} & \mathrm{M} & \mathrm{N} & \mathrm{O} & \mathrm{P} \\
\hline
\end{array}
$$

\begin{solution}

如明文:Person。加密后可以得到:FUHIED。

\end{solution}
\end{example}


由此我们可以知道,密钥空间$|K| = $密钥数量$ = 26$。

或者$|K| = 25$,当偏移量为26时,字母A替换成字母A这种情况不算。


\begin{example}

使用$Python$程序加密I am a person.


\begin{solution}
\begin{lstlisting}
Plaintext = 'iamaperson'
ciphertext = ''

key = 4 %偏移量

for i in Plaintext:
    if (ord(i) + key) > 122:
            alpha =  chr((ord(i)  + key)  - 26)
    else:
        alpha = chr(ord(i) + key)
    ciphertext += alpha
print(ciphertext)
\end{lstlisting}

输出:meqetivwsr 


\end{solution}

\end{example}
 

作为明文时一般没有空格,大小写也不区分。


\section{解密}
由于\textbf{拉丁字母表}中的字母有且仅有26个,使得凯撒密码易受频率分析和暴力破解的攻击。
最多26种可能性即可破解。

\begin{example}
破解密文:exxegoexsrgi
\begin{solution}
~\\

使用$Python$程序破解:
\begin{lstlisting}
cipher = 'exxegoexsrgi'

for j in range(26):
    Plaintext = ''
    for i in cipher:
        if (ord(i) - j) < 97:
            alpha =  chr((ord(i) - j)  + 26)
        else:
            alpha = chr(ord(i) - j)
        Plaintext += alpha
    print(Plaintext)
\end{lstlisting}

输出:
$$
\begin{array}{|c|c|}
\hline \text {Shift} & \text {Output} \\
\hline A & \text {\color{red} exxegoexsrgi} \\
\hline B & \text {dwwdfndwrqfh} \\
\hline C & \text {cvvcemcvqpeg} \\
\hline D & \text {buubdlbupodf} \\
\hline E & \text {\color{blue} attackatonce} \\
\hline F & \text {zsszbjzsnmbd} \\
\hline \ldots & \ldots \\
\hline X & \text {haahjrhavujl} \\
\hline Y & \text {gzzgiqgzutik} \\
\hline Z & \text {fyyfhpfytshj} \\
\hline
\end{array}
$$

由此可知,解密后明文为attack at once

\end{solution}
\end{example}


\section{凯撒密码的改进}
是否觉得凯撒密码太过于简单被破解?我们有三种方法可以使得密文更难破解:
\begin{itemize}
\end{itemize}
\begin{itemize}
\item 随机替换(Randomize the Order of Substitution)
\item 特殊符号替代(Homophonic Substitution)
\item 多字母替代(Poly-alphabetic Substitution)
\end{itemize}

\subsection{随机替换}
即不按照字母表顺序进行替换,而是随机移位,使得每个字母一一对应。如:
$$
\begin{array}{|l|l|l|l|l|l|l|l|l|l|l|}
\hline \mathrm{a} & \mathrm{b} & \mathrm{c} & \mathrm{d} & \mathrm{e} & \mathrm{f} & \mathrm{g} & \mathrm{h} & \mathrm{i} & \mathrm{j} & \ldots \\
\hline \mathrm{c} & \mathrm{t} & \mathrm{h} & \mathrm{a} & \mathrm{q} & \mathrm{z} & \mathrm{x} & \mathrm{v} & \mathrm{n} & \mathrm{p} & \ldots \\
\hline
\end{array}
$$	

计算密钥空间。我们知道‘A’的偏移选择一共有26种(包括自己),‘B’的偏移选择一共有25种,以此列推。密钥空间一共就有$26!$种,介于$2^{88}$和$2^{89}$之间。相比起传统的凯撒密码,破解难度大大提升。

然而,该方法还是易受频率分析的影响而被破解。明文和密文仍然存在遵循字母的频率分布。即统计一串字符每个字母出现次数除以总字符数(仅字母)。
\begin{figure}[H] %H为当前位置,!htb为忽略美学标准,htbp为浮动图形
\centering %图片居中
\includegraphics[width=0.7\textwidth]{图库/1.2.png} %插入图片,[]中设置图片大小,{}中是图片文件名
\caption{英语语言材料中的字母频率} %最终文档中希望显示的图片标题
\label{Fig.main2} %用于文内引用的标签
\end{figure}

明文足够情况下,通过对密文的出现字母频率分析,依然可以推理出明文。

\clearpage

\subsection{特殊符号替代}

字母可以使用特殊符号替代,标点符号,数字,数学符号,希腊字母表,甚至是emoji表情都可以。并且每个字母可以分配多个密文符号。

\begin{figure}[H] %H为当前位置,!htb为忽略美学标准,htbp为浮动图形
\centering %图片居中
\includegraphics[width=0.7\textwidth]{图库/1.3.png} %插入图片,[]中设置图片大小,{}中是图片文件名
\caption{特殊符号替代} %最终文档中希望显示的图片标题
\label{Fig.main2} %用于文内引用的标签
\end{figure}

\begin{example}

密文$\Diamond\Box \Box \Delta $既可以是$FOOD$也可以是 $AMMO$。 $\hfill\square$ 

\end{example}

\begin{example}

明文$OK$既可以是 “$83$” 也可以是 “$[3$”。 $\hfill\square$ 

\end{example}

相比随机替换,不容易被频率分析破解。


\clearpage

\subsection{多字母替代}
1467年, 莱昂·巴蒂斯塔·阿尔贝蒂(Leon Battista Alberti)发明了密码盘。允许发送者对明文的不同部分使用不同的字母。16世纪,有人根据给定的密钥使用多个凯撒密码对明文进行加密
。

~\\

\begin{example}

$$
\begin{aligned}
&\begin{array}{|c|c|c|c|c|c|c|c|c|c|c|c|c|c|}
\hline \mathrm{A} & \mathrm{B} & \mathrm{C} & \mathrm{D} & \mathrm{E} & \mathrm{F} & \mathrm{G} & \mathrm{H} & \mathrm{I} & \mathrm{J} & \mathrm{K} & \mathrm{L} & \mathrm{M} & \mathrm{N} \\
\hline \mathrm{\color{red}R} & \mathrm{S} & \mathrm{T} & \mathrm{U} & \mathrm{V} & \mathrm{W} & \mathrm{X} & \mathrm{Y} & \mathrm{Z} & \mathrm{A} & \mathrm{B} & \mathrm{C} & \mathrm{D} & \mathrm{E} \\
\hline \mathrm{\color{red}E} & \mathrm{F} & \mathrm{G} & \mathrm{H} & \mathrm{I} & \mathrm{J} & \mathrm{K} & \mathrm{L} & \mathrm{M} & \mathrm{N} & \mathrm{O} & \mathrm{P} & \mathrm{Q} & \mathrm{R} \\
\hline \mathrm{\color{red}V} & \mathrm{W} & \mathrm{X} & \mathrm{Y} & \mathrm{Z} & \mathrm{A} & \mathrm{B} & \mathrm{C} & \mathrm{D} & \mathrm{E} & \mathrm{F} & \mathrm{G} & \mathrm{H} & \mathrm{I} \\
\hline
\end{array}\\
&\begin{array}{|c|c|c|c|c|c|c|c|c|c|c|c|}
\hline \mathrm{O} & \mathrm{P} & \mathrm{Q} & \mathrm{R} & \mathrm{S} & \mathrm{T} & \mathrm{U} & \mathrm{V} & \mathrm{W} & \mathrm{X} & \mathrm{Y} & \mathrm{Z} \\
\hline \mathrm{F} & \mathrm{G} & \mathrm{H} & \mathrm{I} & \mathrm{J} & \mathrm{K} & \mathrm{L} & \mathrm{M} & \mathrm{N} & \mathrm{O} & \mathrm{P} & \mathrm{Q} \\
\hline \mathrm{S} & \mathrm{T} & \mathrm{U} & \mathrm{V} & \mathrm{W} & \mathrm{X} & \mathrm{Y} & \mathrm{Z} & \mathrm{A} & \mathrm{B} & \mathrm{C} & \mathrm{D} \\
\hline \mathrm{J} & \mathrm{K} & \mathrm{L} & \mathrm{M} & \mathrm{N} & \mathrm{O} & \mathrm{P} & \mathrm{Q} & \mathrm{R} & \mathrm{S} & \mathrm{T} & \mathrm{U} \\
\hline
\end{array}
\end{aligned}
$$

~\\

其中,密钥(Key)为${\color{red}REV}$


对明文${\color{red}Avenged\ Sevenfold}$进行加密:

$$
\begin{aligned}
&\begin{array}{|c|c|c|c|c|c|c|c|c|c|c|c|c|c|c|c|c|}
\hline \mbox{明文}&\mathrm{A} & \mathrm{V} & \mathrm{E} & \mathrm{N} & \mathrm{G} & \mathrm{E} & \mathrm{D} & \mathrm{S} & \mathrm{E} & \mathrm{V} & \mathrm{E} & \mathrm{N} & \mathrm{F}& \mathrm{O}& \mathrm{L}& \mathrm{D} \\
\hline  Key 1&{\color{red}R}& && {\color{red}E} && &{\color{red}U}& &&{\color{red}M} && &{\color{red}W}& &&{\color{red}U}\\
\hline  Key 2&& {\color{blue}Z}&& &{\color{blue}K}& &&{\color{blue}W} && &{\color{blue}I}& &&{\color{blue}S} &&\\
\hline  Key 3&& &{\color{green}Z}& &&{\color{green}Z} && &{\color{green}Z}& &&{\color{green}I} && &{\color{green}G}&\\
\hline  \mbox{密文}&R&Z&Z&E&K&Z&U&W&Z&M&I&I&W&S&G&U\\
\hline 
\end{array}\\
\end{aligned}
$$

得到密文$RZZEKZUWZMIIWSGU$

\end{example}

$\hfill\square$ 
 
~\\

多字母替代的密钥空间等于26的密钥长度次方。如果Key = WATER, 则密钥空间为$26^{5}$。

其安全性再被发明后的三个世纪内都没有被破解。直至1863年被查尔斯·巴贝奇(Charles Babbage)使用Kasiski's 测试
,利用推断Key的长度,破解多字母替代加密法。












\end{document}

