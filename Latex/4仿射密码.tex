\documentclass{article}
\usepackage[UTF8]{ctex}
\usepackage[utf8]{inputenc} % allow utf-8 input
\usepackage[T1]{fontenc}    % use 8-bit T1 fonts
\usepackage{hyperref}       % hyperlinks
\usepackage{url}            % simple URL typesetting
\usepackage{booktabs}       % professional-quality tables
\usepackage{amsfonts}       % blackboard math symbols
\usepackage{nicefrac}       % compact symbols for 1/2, etc.
\usepackage{microtype}      % microtypography
\usepackage{lipsum}
\usepackage{geometry}
\usepackage{graphicx} %插入图片的宏包
\usepackage{float} %设置图片浮动位置的宏包
\usepackage{subfigure} %插入多图时用子图显示的宏包
\usepackage{listings}

\geometry{a4paper,scale=0.8}
\date{}

\usepackage{listings}
\usepackage{color}

\definecolor{dkgreen}{rgb}{0,0.6,0}
\definecolor{gray}{rgb}{0.5,0.5,0.5}
\definecolor{mauve}{rgb}{0.58,0,0.82}

\lstset{frame=tb,
  language=Python,
  aboveskip=3mm,
  belowskip=3mm,
  showstringspaces=false,
  columns=flexible,
  basicstyle={\small\ttfamily},
  numbers=none,
  numberstyle=\tiny\color{gray},
  keywordstyle=\color{blue},
  commentstyle=\color{dkgreen},
  stringstyle=\color{mauve},
  breaklines=true,
  breakatwhitespace=true,
  tabsize=3
}

\title{仿射密码}


\author{
Affine Cipher\\
 刘卓\\
 \texttt{ } \\
}

\begin{document}
\maketitle

\section{模逆元}
\textit{模逆元(Modular Multiplicative Inverse)。 如果存在$A \cdot C = 1 \bmod B$, ,则C
是B模下A的模逆。 即: $C = A^{-1} \bmod B$。}
\textit{
如果$A^{-1}$存在,则A在B模下是可逆的。
}

\textbf{例1}

令$B = 26$,找到26以内的所有A逆。

模26的逆:

$$
\begin{array}{cc}
 {1^{-1} \equiv 1(\bmod 26)} & 3^{-1} \equiv 9(\bmod 26) \\
 9^{-1} \equiv 3(\bmod 26) & 5^{-1} \equiv 21(\bmod 26) \\ 
 21^{-1} \equiv 5(\bmod 26) & 7^{-1} \equiv 15(\bmod 26) \\ 
 15^{-1} \equiv 7(\bmod 26) & 11^{-1} \equiv 19(\bmod 26) \\
 19^{-1} \equiv 11(\bmod 26) & 17^{-1} \equiv 23(\bmod 26)\\
 23^{-1} \equiv 17(\bmod 26) & 25^{-1} \equiv 25(\bmod 26)\\
\end{array}
$$

其他26以内的数不存在。

$\hfill\square$ 

\section{最大公约数}
最大公约数(greatest common divisor)。让$d,n$为整数。当d除n或者n被d除时,表示为$d|n$,读作“d 整除n”。如果存在一个整数$r$, 使得$n=d \cdot r$。所以d是n的约数(divisor)。

设两个非零整数$m,n$,他们的共同约数是正整数$d$,则$d|m$ 和 $d|n$。

他们的最大公约数(GCD)拥有一个共同约数$d$,并使得d大于其他约数。记作$d = gcd(m,n)$。

如果$gcd(m,n) = 1$,则称$m,n$互质(coprime)。

当且仅当$gcd(a,n) = 1$时,a在n模下是可逆的。


\textbf{例2}

寻找60和42的最大公约数。

解:

$$60 = 2^{2} + 3^{1} + 5^{1}$$

$$42 = 2^{1} + 3^{1} + 7^{1}$$

使用质因数分解法,gcd是所有共同的质数和最小的次方相乘。
$gcd = 2^{1} + 3^{1} = 6$

\section{仿射密码}
仿射密码(Affine cipher)是一种替换密码。它需要几个条件:
\begin{enumerate}
\item a和b是整数;
\item a和26互质,即$gcd(a,26)=1$;
\item $0 \leq b\leq 25$;
\end{enumerate}

加密过程是$y=E(x)=a x+b \bmod 26 $

解密过程是$x=D(y)=a^{-1}(y-b) \bmod 26$

$\hfill\square$ 

~\\

\textbf{例3}

加密明文$SWORD$,令$a=9,b=15$

解:

\setlength{\tabcolsep}{1mm}{
\begin{tabular}{|c|c|c|c|c|c|c|c|c|c|c|c|c|c|c|c|c|c|c|c|c|c|c|c|c|c|}% 通过添加 | 来表示是否需要绘制竖线
\hline  % 在表格最上方绘制横线
0&1&2&3&4&5&6&7&8&9&10&11&12&13&14&15&16&17&18&19&20&21&22&23&24&25\\
\hline  %在第一行和第二行之间绘制横线
A&B&C&D&E&F&G&H&I&J&K&L&M&N&O&P&Q&R&S&T&U&V&W&X&Y&Z\\
\hline % 在表格最下方绘制横、
\end{tabular}
}


$$
\begin{array}{|c|c|c|c|c|c|}
\hline \text { \mbox{明文} } & \mathrm{S} & \mathrm{W} & \mathrm{O}  & \mathrm{N} & \mathrm{D} \\
\hline x & 18 & 22 & 14 & 17 & 3 \\
\hline y=ax+b \bmod 26 & {\color{blue}21} & {\color{blue}5} & {\color{blue}11} & {\color{blue}12} & {\color{blue}16} \\
\hline \text { \mbox{密文} } &  {\color{red}\mathrm{V}} &  {\color{red}\mathrm{F}} &  {\color{red}\mathrm{L}} &  {\color{red}\mathrm{M}} &  {\color{red}\mathrm{Q}} \\
\hline
\end{array}
$$

$\hfill\square$ 


~\\

\textbf{例4}

解:

解密$SYLNH$,令$a=19,b=13$

$a^{-1} = 11$

$$
\begin{array}{|c|c|c|c|c|c|}
\hline \text { \mbox{明文} }  & \mathrm{S} & \mathrm{Y} & \mathrm{L} & \mathrm{N} & \mathrm{H} \\
\hline y & 18 & 24 & 11 & 13 & 7 \\
\hline x=a^{-1}(y-b) \bmod 26 & {\color{blue}3} & {\color{blue}17} & {\color{blue}4} & {\color{blue}0} & {\color{blue}12} \\
\hline  \text { \mbox{密文} }  & {\color{red}D} & {\color{red}R} & {\color{red}E} & {\color{red}A} & {\color{red}N} \\
\hline
\end{array}
$$

$\hfill\square$ 


\clearpage

\textbf{例5}

加密I CAN PLAY,令$a=7,b=25$
\begin{lstlisting}
Plaintext = 'ICANPLAY'#字母需要大写
a = 7
b = 25

Ciphertext = ''
for i in Plaintext:
    x = ord(i) -65 
    y = (a*x + b)%26
    Ciphertext += chr(y+ 65) 
\end{lstlisting}

输出:DNZMAYZL

$\hfill\square$ 


~\\

\textbf{例6}

如果已知明文开头是$GO$,并用仿射密码加密。破解密文$EKTWQMRVRVWQMTF$。

解:

已知:
$$
\begin{array}{l}
G \rightarrow E \\
O \rightarrow K
\end{array}
$$

$G = 6,O = 14$为$x$, 代入$ax+b=y \bmod 26$,等于$y,4=E,10=K$,解二元一次方程。

$$
\left\{\begin{array}{l}
6 a+b=4\\
14 a+b=10
\end{array}\right \bmod 26 .
$$

$$
\left\{\begin{array}{l}
6 a+b=4\\
14 a+b=10
\end{array}\right \bmod 26 \quad \Rightarrow \quad
\left\{\begin{array}{l}
a = 4 \\
b= 6 
\end{array}\right \bmod 26 \quad or 
\left\{\begin{array}{l}
a = 17 \\
b= 6 
\end{array}\right \bmod 26
$$

然后$gcd(4,6) \ne 1, (4,6)$不是互质,而$gcd(17,6) =1$,所以$a= 17,b=6$。

根据$x=a^{-1}(y-b) \bmod 26$,明文为GONE WITH THE WIND


$\hfill\square$ 






















\end{document}

