\documentclass{article}
\usepackage[UTF8]{ctex}
\usepackage[utf8]{inputenc} % allow utf-8 input
\usepackage[T1]{fontenc}    % use 8-bit T1 fonts
\usepackage{hyperref}       % hyperlinks
\usepackage{url}            % simple URL typesetting
\usepackage{booktabs}       % professional-quality tables
\usepackage{amsfonts}       % blackboard math symbols
\usepackage{nicefrac}       % compact symbols for 1/2, etc.
\usepackage{microtype}      % microtypography
\usepackage{lipsum}
\usepackage{geometry}
\usepackage{graphicx} %插入图片的宏包
\usepackage{float} %设置图片浮动位置的宏包
\usepackage{subfigure} %插入多图时用子图显示的宏包
\usepackage{listings}
\usepackage{soul}


\geometry{a4paper,scale=0.8}
\date{}

\usepackage{listings}
\usepackage{color}

\definecolor{dkgreen}{rgb}{0,0.6,0}
\definecolor{gray}{rgb}{0.5,0.5,0.5}
\definecolor{mauve}{rgb}{0.58,0,0.82}

\lstset{frame=tb,
  language=Python,
  aboveskip=3mm,
  belowskip=3mm,
  showstringspaces=false,
  columns=flexible,
  basicstyle={\small\ttfamily},
  numbers=none,
  numberstyle=\tiny\color{gray},
  keywordstyle=\color{blue},
  commentstyle=\color{dkgreen},
  stringstyle=\color{mauve},
  breaklines=true,
  breakatwhitespace=true,
  tabsize=3
}

\title{维吉尼亚密码及其破译}


\author{
Breaking Vigenère Cipher\\
 刘卓\\
 \texttt{ } \\
}

\begin{document}
\maketitle

维吉尼亚密码(Breaking Vigenère Cipher)是使用一系列凯撒密码组成密码字母表的加密算法,即多字母替代凯撒密码。属于多表密码的一种简单形式。


\section{排列}
排列(Permutation),是将相异对象或符号根据确定的顺序重排。每个顺序都称作一个排列。即:从n个对象取k个对象的有序排列,其中$k \leq n$。一共有$$
P(n, k):=n \cdot(n-1) \cdot(n-2) \cdots(n-k+2) \cdot(n-k+1)=\frac{n !}{(n-k) !}
$$种方法.

\section{组合}
组合(Combination),一个集的元素的组合是一个子集。若两个子集的元素完全相同并顺序相异,它仍视为同一个组合。即:从n个对象取k个对象的无序排列,其中$k \leq n$。读作n取k。$$
C(n, k)=\frac{P(n, k)}{k !}=\left(\begin{array}{l}
n \\
k
\end{array}\right)=\frac{n!}{k!(n-k) !}
$$

\section{概率}
对于一项实验,其中存在n种不同的可能性,然后以k种可能的方式出现结果的概率为$\frac{k}{n}$

\textbf{例1}

生日问题。一年有365天,在一个班级中,假设这个班有n个人,求这个班至少有两人在同一天生日的概率。

$$
\begin{aligned}
&1-\frac{P(365, n)}{365^{n}}=1-\frac{365 \cdot 364 \cdots(365-n+1)}{365^{n}}\\
\end{aligned}
$$

$$
&\begin{array}{|c|c|c|c|c|c|c|c|c|}
\hline n & 1 & 2 & 3 & 10 & 20 & 30 & 40 & 50 \\
\hline p & 0 & 0.0027 & 0.0082 & 0.117 & 0.411 & 0.706 & 0.891 & 0.97 \\
\hline
\end{array}
$$

即代表如果一个班有50个人,基本可以确定至少有两人在同一天生日。

$\hfill\square$ 

\textbf{例2}

假设甲板上有45张卡。 在这些卡中,有20张标有“ X”,15张标有“ Y”,以及10张“ Z”。随机选择一张牌,放回去,然后随机洗牌并随机选择另一张牌。 

(1)找出第一张是X第二张是Z的概率。

$$
P(X \ \text { and } \ Z)=\mathbb{P}(x) \cdot \mathbb{P}(z)=\frac{20}{45} \cdot \frac{10}{45}=\frac{8}{81}
$$

(2)两张卡分别是X和Z

$$
\begin{aligned}
\mathbb{P}\left(X \ \text {and } \ Z\right) &=\mathbb{P}(X \ \text { and } \ Z)+\mathbb{P}(Z \ \text { and } \ X) \\
&=\frac{8}{81}+\frac{8}{81}=\frac{16}{81}
\end{aligned}
$$

(3) 两张都是Y

$$
\mathbb{P}(Y \ \text { and} \ Y)=\mathbb{P}(Y) \cdot \mathbb{P}(Y)=\frac{15}{45} \cdot \frac{15}{15}=\frac{1}{9}
$$ 

$\hfill\square$ 




下表列出了7834个字母的英语写作样本中的字母的相对频率。

$$
\begin{array}{|c|c||c|c|}
\hline \text { Letter } & \text { Relative  frequency (\%) } & \text { Letter } & \text { Relative  frequency (\%) } \\
\hline \text { A } & 8.399 & \text { N } & 6.778 \\
\hline \text { B } & 1.442 & \text { O } & 7.493 \\
\hline \text { C } & 2.527 & \text { P } & 1.991 \\
\hline \text { D } & 4.800 & \text { Q } & 0.077 \\
\hline \text { E } & 12.150 & \text { R } & 6.063 \\
\hline \text { F } & 2.132 & \text { S } & 6.319 \\
\hline \text { G } & 2.323 & \text { T } & 8.999 \\
\hline \text { H } & 6.025 & \text { U } & 2.783 \\
\hline \text { I } & 6.485 & \text { V } & 0.996 \\
\hline \text { J } & 0.102 & \text { W } & 2.464 \\
\hline \text { K } & 0.689 & \text { X } & 0.204 \\
\hline \text { L } & 4.008 & \text { Y } & 2.157 \\
\hline \text { M } & 2.566 & \text { Z } & 0.025 \\
\hline
\end{array}
$$

随机取两个字母,其两个字母相同的概率为$\mathbb{P}(2 Letter) = \mathbb{P}(2A)+ \mathbb{P}(2B) + \cdots +\mathbb{P}(2Z)=0.08399^{2}+0.01442^{2}+ \cdots +0.00025^{2}=6.5\% $

~\\

\textbf{例3}

1.如果\textbf{凯撒密码}转化字符的方法为$A→D$,密文中随机选择一个字母A的概率是多少?

$$\mathbb{P}(\mbox{密文中的}A)  = \mathbb{P}(\mbox{明文中的}X) = \mathbb{P}(X) = 0.204\%$$

2. 密文中随机选择一个字母B的概率是多少?
$$\mathbb{P}(\mbox{密文中的}B)  = \mathbb{P}(\mbox{明文中的}Y) = \mathbb{P}(Y) = 2.157\% $$

$\hfill\square$ 



\textbf{例4}
假设使用维吉尼亚密码(Vigenère Cipher),密钥(Key)是$DN$。

(1)密文中A的概率?
$$\mathbb{P}(\mbox{密文中的}A)  = \mathbb{P}(\mbox{明文中奇数位是}X) + \mathbb{P}(\mbox{明文中偶数位是}N)  = \frac{\mathbb{P}(X)}{2} + \frac{\mathbb{P}(Y)}{2} = 3.491\%$$

(2)密文中B的概率?
$$\mathbb{P}(\mbox{密文中的}B)  = \mathbb{P}(\mbox{明文中奇数位是}Y) + \mathbb{P}(\mbox{明文中偶数位是}O)  = \frac{\mathbb{P}(Y)}{2} + \frac{\mathbb{P}(O)}{2} = 4.825\%$$


$\hfill\square$ 



~\\

以此列推,即使使用长密钥,在密文中看到任何字母的概率将收敛为$\frac{1}{26}$

\textbf{条件概率}是指事件A已经发生了,发生事件B的概率。
$$\mathbb{P}(B|A) = \frac{\mathbb{P}(B \mbox{和} A)}{\mathbb{P}(A)}$$

\section{重合因子}
重合因子(Index of Coincidence)是密文中两个随机选择的字母相同的概率,记为$I$。

\begin{itemize}
\item 如果$I \approx 0.065$,则表示密码很有可能是\textbf{单字母}代替
\item 对于多字母代替,$I$ 的范围是$\frac{1}{26}=0.0385 \leq I \leq 0.065$
\end{itemize}

$$I = \frac{1}{n(n-1)} \sum_{i=0}^{25}n_i(n_i-1), 0 \leq i \leq25$$
其中$n$为文本中所有字符总和。$n_i$为每个字母出现的个数。如A出现五次,记$n_0 = 5$。

如果在拉丁字母表中使用维吉尼亚密码(Vigenère Cipher),其密钥长度为$k$,,为了估计密钥长度,则可以使用弗里德曼检验(Friedman Test):
$$\mbox{重合因子} =  I \approx \frac{0.0385 \times n(k-1)+0.065(n-k)}{k(n-1)}$$
$$\mbox{密钥长度} = k\approx \frac{0.0265n}{(0.065 - I)+n(I - 0.0385)}$$

弗里德曼检验仅仅只能估计密钥长度$k$,但密文长度也\textbf{不能太短}。

\clearpage
\textbf{例5}

已知密文使用维吉尼亚密码(Vigenère Cipher)加密,密文总长为$n=337$,每个字母出现频率如表格所示。估计密钥长度$k$是多少。

解:
$$
\begin{array}{|c|c||c|c|}
\hline \mbox{字母} &\mbox{数量} & \mbox{字母} & \mbox{数量} \\
\hline \text { A } & 13 & \text { N } & 11 \\
\hline \text { B } & 18 & \text { O } & 17 \\
\hline \text { C } & 12 & \text { P } & 21 \\
\hline \text { D } & 15 & \text { Q } & 9 \\
\hline \text { E } & 26 & \text { R } & 16 \\
\hline \text { F } & 4 & \text { S } & 7 \\
\hline \text { G } & 15 & \text { T } & 8 \\
\hline \text { H } & 9 & \text { U } & 7 \\
\hline \text { I } & 16 & \text { V } & 8 \\
\hline \text { J } & 8 & \text { W } & 14 \\
\hline \text { K } & 9 & \text { X } & 8 \\
\hline \text { L } & 18 & \text { Y } & 20 \\
\hline \text { M } & 22 & \text { Z } & 6 \\
\hline
\end{array}
$$

$$I = \frac{1}{n(n-1)} =  \sum_{i=0}^{25}n_i(n_i-1) = \frac{1}{337\times336}[13 \times 12 + 18 \times 17 + \cdots +6\times 5 = 0.0428$$
$$k  = \frac{0.0265\times 337}{(0.065-0.0428)+337 \times (0.0428-0.0385)}  \approx6.2\approx6$$


$\hfill\square$ 


\section{卡西斯基检验}
卡西斯基检验(Kasiski Test)是另一种维吉尼亚密码(Vigenère Cipher)中估算密钥长度的方法。它从密文中重复字母组之间的最大公约数(gcd)中获得可能的密钥长度。

\textbf{例6}

估算密文$I V E V Y G A R M L M Y I V E K F D I V E F R L$密钥长度

解:

$$ \underbrace{{\color{red}IVE}V Y G A R M L M Y }_{12}  \underbrace{{\color{red}IVE} K F D}_{6} \underbrace{{\color{red}IVE} FRL}_{6}$$
$$k \approx gcd(12,6) = 6$$


$\hfill\square$ 


\section{密码分析}
弗里德曼检验(Friedman Test)和卡西斯基检验(Kasiski Test)只能估计密钥长度,而不能直接猜出密钥本身。而且有一定局限性,通常少于400个字符时,检验的准确率不高。

















































\end{document}

